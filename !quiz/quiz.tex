\documentclass[12pt]{article}
\usepackage[utf8]{inputenc}
\usepackage{amsmath,amssymb}
\usepackage{unicode-math}
\usepackage[T2A]{fontenc}
\usepackage[russian]{babel}
\usepackage{graphicx}
\usepackage{subfigure}
\usepackage{subcaption}
\usepackage{url}


\DeclareGraphicsExtensions{.pdf,.png,.jpg}
\usepackage{hyperref}
\usepackage{wrapfig}
\usepackage[left=20mm, top=20mm, right=10mm, bottom=20mm]{geometry}

\usepackage{amsmath} 
\usepackage{amsfonts} 
\usepackage{amssymb} 
\usepackage{wasysym} 
\usepackage{fancyhdr}



\begin{document} 
\title{\textbf{Список вопросов по курсу "Квантовая микрофизика" }}
\author{\textbf{Клименок Кирилл Леонидович}}
\date{01.09.2020}
\maketitle
\section*{Вопросы по семинарам}
\subsection*{Семинар 1. Излучение}
\begin{itemize}
    \item Определение и примеры равновесного теплового излучения. Почему оно равновесное?
    \item Абсолютно черное тело. Что это такое? Какие у него отражательная и поглощательная способность?
    \item Почему для любого тела отражательная способность равна поглощательной? (Ответ в лекции Глазкова)
    \item Модель АЧТ. Связь энергии и спектральной плотности излучения. Давление излучения
    \item Ультрафиолетовая катастрофа. В чем проблема?
    \item Гипотеза Планка, отличие от классики
    \item Формула Планка. Написать в зависимости от частоты и длины волны. Объяснить различия в степенях
    \item Закон смещения Вина, закон Стефана-Больцмана. 
\end{itemize}
\subsection*{Семинар 2. Фотоэффект. Эффект Комптона}
\begin{itemize}
    \item Экспериментальные результаты по фотоэффекту. Что наблюдается?
    \item Красная граница фотоэффекта. Что это, о чем свидетельствует?
    \item Нарисовать вольт-амперные характеристики для разных интенсивностей (отвечает за полку) и разных длин волн света (отвечает за запирающее напряжение)
    \item Закон Эйнштейна для фотоэффекта. Физический смысл работы выхода
    \item Суть эффекта Комптона. Характерные длины волн света в эффекте
    \item Вывод комптоновской длины волны и формулы рассеяния (идея)
\end{itemize}
\subsection*{Семинар 3. Соотношение неопределенностей}
\begin{itemize}
    \item Корпускулярно-волновой дуализм. Связь импульса с длиной волны и волновым вектором. 
    \item Опыты по дифракции и интерференции электронов. Суть явления.
    \item Вывод соотношения неопределенностей
    \item Почему в квантовом осцилляторе  все равно есть колебания, даже когда энергии нет
    \item Оценка уровней энергии в атоме водорода из соотношения неопределенностей
    \item Парадокс Эйнштейна-Подольского-Розена. \textit{Примечание:} прочтите что это такое, опишите этот парадокс студентам и предложите им решить его. Потом расскажите про квантовую запутанность 
\end{itemize}

\subsection*{Семинар 4. Волновая функция. Операторы. Уравнение Шредингера. Барьеры}
\begin{itemize}
    \item Связь длины волны и импульса. Общий вид волновой функции и ее физический смысл. 
    \item Проблема нормировки волновой функции вида "плоская волна"
    \item Свойства волновой функции (непрерывность, гладкость)
    \item Операторы физических величин. Смысл, связь наблюдаемых величин и собственных значений оператора
    \item Уравнение Шредингера в стационарном и нестационарном виде
    \item Туннелирование. Коэффициент прохождения через потенциальный барьер произвольной формы
    \item Поток плотности вероятности. Как он получается, для чего он нужен.
\end{itemize}

\subsection*{Семинар 5. Волновая Функция. Потенциальные ямы. Уровни энергии. Квазиклассика}
\begin{itemize}
    \item Уровни энергии в потенциальной яме с бесконечными стенками.
    \item Волновые функции в потенциальной яме с бесконечными стенками.
    \item Эффект Рамзауэра. Смысл.
    \item Квазиклассическое приближение. Когда и как работает?
    \item Прикол с трехмерной и одномерной ямами (при встрече напомню)
\end{itemize}

\subsection*{Семинар 6. Уровни энергии в водородоподобном атоме.}
\begin{itemize}
    \item Оператор момента импульса. Почему мы не можем использовать его однозначно.
    \item Собственные значения оператора проекции момента импульса на выделенную ось, откуда получается.
    \item Квадрат момента импульса. Вращательные уровни энергии. Температура возбуждения вращательных уровней энергии.
    \item Уровни энергии в водородоподобном атоме. Постоянная Ридберга. Изотопический сдвиг.
    \item Электронные орбитали. Квантовые числа.
\end{itemize}

\subsection*{Семинар 7. Спин. Сложный атом}
\begin{itemize}
    \item Классическая связь магнитного и механического моментов. Магнетон Бора.
    \item Опыт Эйнштейна - де Гааза. Как он подтверждает эту связь?
    \item Опыт Штерна-Герлаха. Как он устроен. Почему не складывается с предыдущей теорией (и классической и квантовой).
    \item Полный момент $J$ и полный магнитный момент $\mu_J$. Как связаны? Причем тут g-фактор?
    \item Запись терма атома. Пусть расскажут как в задаче T2 перебирали варианты и получили ответ.
    \item Принцип Паули.
\end{itemize}

\subsection*{Семинар 8. Эффект Зеемана. Правила отбора}
\begin{itemize}
    \item Четность. Что это, собственные значения оператора четности.
    \item Различие нормальных и аксиальных векторов при инверсии
    \item Золотое правило Ферми, просто суть
    \item Мультипольное разложение, E/M фотоны
    \item Зееман в сильном и слабом поле
    \item Суть ЯМР/ЭПР, основная формула эффекта
\end{itemize}

% \subsection*{Семинар }
% \begin{itemize}
%     \item 
%     \item 
%     \item 
%     \item 
%     \item 
%     \item 
% \end{itemize}

\end{document}
