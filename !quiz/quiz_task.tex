\documentclass[11pt,a4paper]{article}
\usepackage[utf8]{inputenc}
\usepackage{amsmath}
\usepackage{amsfonts}
\usepackage{amssymb}
\usepackage{makeidx}
\usepackage{graphicx}
\usepackage[russian]{babel}
\usepackage[left=2cm,right=2cm,top=2cm,bottom=2cm,bindingoffset=0cm]{geometry}
\author{Клименок Кирилл Леонидович}
\title{Контрольная работа по 1 заданию}
\begin{document}
	\pagestyle{empty}
	\section*{Контрольная работа по 1 заданию.\\[10pt] Студент, группа:}
	\begin{tabular}{|p{2cm}|p{1cm}|p{1cm}|p{1cm}|p{1cm}|p{1cm}|p{1cm}|}
		\hline \rule[-2ex]{0pt}{5.5ex} № Задачи & 1 & 2 & 3 & 4 & 5 & Итого \\ 
		\hline \rule[-2ex]{0pt}{5.5ex} Баллы &  &  &  &  &  &  \\ 
		\hline 
	\end{tabular} 
	\\
	\fbox{
	\begin{minipage}{\textwidth} 
	\centering\textbf{Константы}\\
	$\hbar = 1.05\cdot 10^{-34} \text{ Дж}\cdot\text{c} = 6.6\cdot10^{-16} \text{ эВ}\cdot\text{c}$, $m_e =9.1\cdot 10^{-31}$ кг, $m_p=m_n=1836 m_e$ $\sigma = 5.67 \cdot 10^{-8} \dfrac{\text{Вт}}{\text{м}^2 \text{К}^4}$, $k_B=1.38 \cdot 10^{-23} \dfrac{\text{Дж}}{\text{К}} = 8.62 \cdot 10^{-5} \dfrac{\text{эВ}}{\text{К}}, G= 6.67\cdot10^{-11} \dfrac{\text{Н}\cdot\text{м}^2}{\text{кг}^2}$
	\end{minipage}
	}
	
	\paragraph{Задача 1}(2.5) Молекула азота имеет квант колебательной энергии 0.3 эВ. Оценить долю излучения Солнца (температура поверхности $T_0 = 6000$ К) поглощенную азотом при температуре $300$ К. Считать, что ширина линии определяется эффектом Доплера.
	\paragraph{Задача 2}(2.5) Мяч масс $m$ прыгает вертикально в поле тяжести с ускорением свободного падения $g$, упруго отскакивая от абсолютно твердого пола. Оценить энергию основного состояния и положение уровней энергии $E_n$ при $n \gg 1$.
	\paragraph{Задача 3}(1.5) Электрон с кинетической энергией $K=1$ эВ пролетает над одномерной прямоугольной потенциальной ямой с глубиной $U_0 = -3$ эВ и шириной $a = 6$ \AA. Оценить вероятность прохождения частицы сквозь яму.
	\paragraph{Задача 4}(1.5) Оценить радиус атома водорода, если бы он был обусловлен только гравитационным взаимодействием. \textit{Указание:} использовать модель Бора
	\paragraph{Задача 5}(2) Оценить вращательную теплоемкость $C_V$ атомарного водорода на одну частицу в единицах постоянной Больцмана при комнатной температуре.
	
	\section*{Решение.}
	
\end{document}