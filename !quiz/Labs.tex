\documentclass[12pt]{article}
\usepackage[utf8]{inputenc}
\usepackage{amsmath,amssymb}
\usepackage{unicode-math}
\usepackage[T2A]{fontenc}
\usepackage[russian]{babel}
\usepackage{graphicx}
\usepackage{subfigure}
\usepackage{subcaption}
\usepackage{url}


\DeclareGraphicsExtensions{.pdf,.png,.jpg}
\usepackage{hyperref}
\usepackage{wrapfig}
\usepackage[left=20mm, top=20mm, right=10mm, bottom=20mm]{geometry}

\usepackage{amsmath} 
\usepackage{amsfonts} 
\usepackage{amssymb} 
\usepackage{wasysym} 
\usepackage{fancyhdr}



\begin{document} 
% \title{\textbf{Список вопросов по курсу "Квантовая микрофизика" }}
% \author{\textbf{Клименок Кирилл Леонидович}}
% \date{01.09.2020}
% \maketitle
\section*{Вопросы по Лабораторным работам}
% \subsection*{Работа 7.1/7.4 Космическое излучение}
% \begin{itemize}
%     \item Космические лучи. Что это из чего состоят. Характерные энергии.
%     \item Методы регистрации частиц. Газовый счетчик. Сцинтиллятор. Фотоэлектронный умножитель. Что у нас в установке.
%     \item Сечение ядерных реакций. Резонансный и не резонансный случаи. Закон Бете.
%     \item Элементарные частицы. Стандартная модель
% \end{itemize}

% \subsection*{Работа 1.3 Эффект Рамзауэра и 2.3 Опыт Франка-Герца}
% \begin{itemize}
%     \item Устройство вакуумной лампы/тиратрона.
%     \item Структура вольт-амперной характеристики. Смысл локальных максимумов и минимумов.
%     \item Эффект Рамзауэра. Суть, коэффициенты отражения и прохождения в одномерной модели.
%     \item Опыт Франка-Герца. Суть. Отличия от эффекта Рамзауэра
%     \item Волновая функция. Свойства и смысл. 
%     \item Уравнение Шредингера. Решение для потенциальной ямы с бесконечными стенками.
% \end{itemize}

\subsection*{Работа 4.1 Определение энергии $\alpha$-частиц}
\begin{itemize}
    \item Устройство схемы с счетчиком Гейгера. Смысл графика $N(h)$ и почему нам нужен именно перегиб
    \item Устройство схемы с ФЭУ. Смысл графика $N(P)$ и почему нам нужен именно перегиб
    \item Схема с ионизационным током. Почему нам нужен коэффициент наклона и откуда появляется полка при больших давлениях
    \item Энергия связи ядра. Удельная энергия связи на нуклон. Как связано с распадами
    \item Почему вылет альфа-частиц из ядра более выгоден, чем вылет отдельных протонов или нейтронов. Энергическая оценка
    \item Закон Гейгера-Неттола
\end{itemize}


\end{document}
