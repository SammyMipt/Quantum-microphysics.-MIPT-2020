\documentclass[12pt]{article}
\usepackage[utf8]{inputenc}
\usepackage{amsmath,amssymb}
\usepackage{unicode-math}
\usepackage[T2A]{fontenc}
\usepackage[russian]{babel}
\usepackage{graphicx}
\usepackage{subfigure}
\usepackage{subcaption}
\usepackage{url}


\DeclareGraphicsExtensions{.pdf,.png,.jpg}
\usepackage{hyperref}
\usepackage{wrapfig}
\usepackage[left=20mm, top=20mm, right=10mm, bottom=20mm]{geometry}

\usepackage{amsmath} 
\usepackage{amsfonts} 
\usepackage{amssymb} 
\usepackage{wasysym} 
\usepackage{fancyhdr}

\pagestyle{fancy}
\fancyhf{}
\lhead{Семинар 3. Соотношение неопределенностей}
\rhead{\textit{Клименок К.Л., МФТИ 2020}}
\rfoot{\thepage}



\begin{document} 
\title{\textbf{Семинар 3. Корпускулярно-волновой дуализм. Соотношение неопределенностей}}
\author{\textbf{Клименок Кирилл Леонидович}}
\date{17.09.2020}
\maketitle
\section{Теоретическая часть}
\subsection{Корпускулярно-волновой дуализм}
На прошлой неделе мы с вами говорили о том, что световая волна может проявлять свойства частицы-фотона и взаимодействовать с другими частицами, например рассеиваться на свободных электронах (эффект Комптона). А может ли это работать и в обратную сторону, когда частица также будет проявлять волновые свойства? Если ответить на этот вопрос положительно, то окажется, что нам можно будет наблюдать такие явления как интерференция и дифракция для "обычных" частиц. Но перед этим я предлагаю оценить масштабы длин волн, с которыми нам придется иметь дело. 
\\
Из прошлых недель мы знаем, что энергия и импульс фотонов связаны следующими соотношениями:
\begin{equation*}
    p = \dfrac{E}{c}; E = \hbar \omega
\end{equation*}
Из этих двух соотношений мы вполне спокойно можем вытащить импульс и по преобразовывать его:
\begin{equation}
    p = \dfrac{\hbar \omega}{c} = \hbar k = \dfrac{h}{\lambda}
\end{equation}
Здесь k -- волновой вектор, $\lambda$ -- длина волны. Вот мы и получили необходимую нам связь. Теперь мы можем оценить характерные длины волн. Возьмем классический пример электрона, разогнанного в поле с напряжением $U=100$ В и подставим в полученное выражение:
\begin{equation*}
    \lambda = \dfrac{h}{m_eV} = \dfrac{h}{\sqrt{2m_eeU}} \approx \dfrac{10^{-34}}{\sqrt{10^{-31}\cdot10^{-19}\cdot10^{2}}} \text{ м} \approx 10^{-10} \text{ м}
\end{equation*}
Характерное число в 1 \AA получается примерно совпадает с длиной волны рентгена, а он, как я надеюсь вы помнить из оптики, отлично дифрагирует на кристаллических решетках твердых тел. Напомню основное соотношение при такой дифракции. Пусть пучек электронов (или рентгеновское излучение) падает на плоскость кристалла под углом скольжения $\theta$, расстояние между такими плоскостям d, тогда направление на максимумы интенсивности определяется соотношением Брэгга-Вульфа: 
\begin{equation*}
    2d\sin{\theta} = n\lambda
\end{equation*}
Такой опыт по дифракции электронов был проведен Дэвиссоном и Джермером, за что в 1937 году Дэвиссону была присуждена Нобелевская премия. \\
В качестве еще одного доказательства волновой природы электрона можно привести эксперименты Hitachi(1989) или Баха(2013),  которые по своей сути повторяли классический опыт Юнга с интерференцией на 2 щелях. Очень слабый ток электронов проходил через специальную фокусирующую систему и отслеживалась точка попадания электрона на экран. Так как ток был очень слабый, можно считать, что электроны проходили через систему по одному. На самом экране в начале эксперимента казалось, что точки попадания случайны, но при накоплении статистики оказывалось, что здесь присутствуют полосы с большей переменной концентрации электронов, то есть опыт Юнга повторялся. И вот тут становится по-настоящему интересно, что происходит и на чем же интерферирует электрон. Вот здесь и заканчивается классическое представление об электроне как о частице с нарисованным минусом и подключается чисто квантовое описание электрона как волнового пакета, который проходя через 2 щели находится в обеих и интерферирует сам с собой. Здесь я не претендую на хорошее и строгое описание, а пытаюсь максимально просто донести саму идею. Ссылка на оригинальную статью Баха с видео формирования интерференционной картины электронов: \url{http://stacks.iop.org/NJP/15/033018/mmedia}
\begin{figure}[h]
    \centering
    \includegraphics[width=\textwidth,height=\textheight,keepaspectratio]{Seminar_03/pics/pic_01.png}
    \caption{Распределение попадания электронов в эксперименте Баха. Каждая точка соответствует задетектированному электрону}
    \label{fig:sem_03_bach_experiment}
\end{figure}
\subsection{Проблема измерения. Соотношение неопределенностей}
С этими новыми "волновыми" электронами мы получили довольно занимательное противоречие нашему предыдущему опыту, как это так, электрон и частица и волна и при этом проявляет только какое-то одно из свойств в зависимости от экспериментов. На деле объяснить это достаточно просто, но нужно пойти к самым истокам нашего изучения физики, а именно к измерению. Рассмотри простой пример измерения положения объекта в пространстве. Для этого нанесем в нашем пространстве сетку координат (или просто положим линейку) и осветим наш объект, чтобы мы могли увидеть куда падает тень и где он относительно линейки. В классической физике и вашем 2-х летнем опыте физического практикума это кажется настолько естественным, что и в случае с отдельным электроном этот трюк вполне можно провернуть. На деле же, как только мы освещаем наш электрон, фотоны сталкивающиеся с ним немного меняют его импульс и он начинает куда-то двигаться. Аналогичная проблема возникает, если мы пытаемся измерить импульс -- становится непонятно где же конкретно этот электрон. \\
\begin{figure}[h]
    \centering
    \includegraphics[width=0.7\textwidth,keepaspectratio]{Seminar_03/pics/pic_02.png}
    \caption{Схема эксперимента по дифракции электронов на щели и для вывода соотношения неопределенностей.}
    \label{fig:sem_03_electron_diffraction}
\end{figure}
Давайте рассмотрим простой пример с дифракцией электрона на отдельной щели определенной ширины. Пусть он летит на щель шириной $d$ с импульсом $p$ перпендикулярно ей. Это означает, что его длина волны $\lambda = h/p$, а из курса оптики, я надеюсь вы помните, что направление на дифракционный минимум зависит от длины волны и ширины щели: $\sin{\theta} = \lambda/d$. То есть у нашего электрона из-за щели появилась "неопределенная" составляющая импульса $\Delta p$, которая его и отклонила от своей начальной траектории. Из геометрии понятно, что $\sin{\theta} = \Delta p / p$ и отсюда окончательно можно сказать, что $d\Delta p = h$. А если мы будем рассматривать нашу щель как попытку локализовать электрон в определенном месте пространства, то ширина щели и будет по суть "неопределенностью" по координате. Окончательно мы можем записать соотношение неопределенностей:
\begin{equation}
    \Delta p_x \Delta x \sim h
\end{equation}
Это соотношение прекрасно подходит для характерных оценок в задачах текущего семинара. Более точная запись соотношения неопределенностей представляет из себя связь дисперсий импульса и координаты и записывается в форме Вейля:
\begin{equation*}
    \sigma^2_p \sigma^2_x \ge \dfrac{\hbar^2}{4}
\end{equation*}
И еще пара ремарок относительно полученных выражений. Во-первых, соотношение неопределенностей напрямую следует из волнового описания частиц и Фурье-анализа, пока я специально не акцентировал на этом внимания и не записывал $\psi$-функцию в явном виде, пока нам это не нужно. Но чтобы освежить в памяти смысл неопределённостей частота-время, я как всегда оставлю ссылку на видео 3Blue1Brown: \url{https://www.youtube.com/watch?v=MBnnXbOM5S4}. Во-вторых, довольно легко математическими преобразованиями получить не только соотношение импульс-координата, но и энергия-время $\Delta E \Delta t \sim h$. Если $\Delta p$ и $\Delta x$ неопределенности в один и тот же момент времени, а в формуле для неопределенности по энергии сама энергия измеряется в разные моменты времени. То есть в действительности присутствует огромная разница между физическими смыслами в двух разных соотношениях неопределенностей —- для энергии и времени, и для
координаты и импульса. Одно из самых ярких примеров использования этого соотношения -- это нарушение закона сохранения энергии на масштабе времен $\Delta t$. Это нам понадобится, когда мы будем говорить о виртуальных частицах.

\section{Практическая часть}
\subsection{Задача 2.15}
\label{task_2.15}
\paragraph{Условие}
Чтобы получить пучок нейтронов обладающих заданной энергией $E = 1$ эВ используют брэгговское отражение первого порядка от кристалла LiF, для которого расстояние между плоскостями решетки $d = 2.32$ \AA/ На кристалл падает пучок нейтронов с различными энергиями. Оценить разброс нейтронов по энергиям в отраженном пучке, если его угловая ширина $\Delta \varphi = 0.1^\{\circ}$. Какую толщину кристалла D следует выбирать в этом эксперименте?
\paragraph{Решение}
На самом деле эта задача больше на оптику, чем на кванты, так что давайте вспоминать все, что мы оттуда помним. Запишем условие Брэгга-Вульфа и оценим масштаб бедствия по углам отклонения:
\begin{gather*}
    \sin{\varphi} = \dfrac{\lambda}{2d}\\
    \lambda = \dfrac{h}{p} = \dfrac{h}{\sqrt{2mE}} \approx 0.287 \text{ \AA}\\
    \sin{\varphi} \approx \varphi \approx 0.06
\end{gather*}
Как и должно было случится, мы получили малые углы. Теперь давайте сориентируемся, как будет влиять изменение угла на изменение энергии. Выше мы увидели следующую закономерность: $\lambda \sim \varphi  \sim E^{-1/2} $ отсюда и будет следовать:
\begin{equation*}
    \dfrac{\Delta \lambda}{\lambda} = \dfrac{\Delta \varphi}{\varphi} = \dfrac{\Delta E}{2E}
\end{equation*}
Выражаем отсюда $\Delta E = 2E \dfrac{\Delta \varphi}{\varphi} \approx 0.58$ эВ\\
Остался не разобранным вопрос с толщиной. Она влияет на количество слоев и, соответственно, на количество отражений. То есть перед нами простая дифракционная решетка с известной нам разрешающей способностью:
\begin{gather*}
    \dfrac{\Delta \lambda}{\lambda} \le R = mN = 1\cdot \dfrac{D}{d}
\end{gather*}
окончательно $D = \dfrac{\lambda}{2\Delta \varphi} = 82 \text{ \AA}$

\subsection{Задача 2.43}
\label{task_2.43}
\paragraph{Условие} Оценить на основании соотношения неопределенностей радиус атома водорода в основном состоянии и энергию связи электрона в том же состоянии. Определить на основании таких же оценок размер двухатомной молекулы и энергию её основного состояния,рассматривая молекулу как одномерный гармонический осциллятор с собственной частотой $\omega_0$ и приведенной массой $\mu$
\paragraph{Решение}
Часть 1. Атом водорода.\\
Для начала надо определиться с тем, что подставлять в соотношение неопределенностей. Тут, так как мы просто оцениваем в качестве неопределенности координаты мы можем взять сам радиус атома, а качестве неопределенности импульса сам импульс: $\Delta x \sim r; \Delta p \sim p \Rightarrow p\cdot r = \hbar $. Теперь запишем полную энергию электрона в атоме водорода:
\begin{gather*}
    E = \dfrac{p^2}{2m} - \dfrac{e^2}{r} = \dfrac{\hbar^2}{2mr^2} - \dfrac{e^2}{r}
\end{gather*}
Атом находится в основном состоянии, значит, в состоянии с минимальной энергией. Найдем минимум через производную энергии по времени. 
\begin{gather*}
    \dfrac{dE}{dt} = -\dfrac{\hbar^22r}{2mr^4} + \dfrac{e^2}{r^2} = 0 \Rightarrow к = \dfrac{\hbar^2}{me^2}=0.53 \text{ \AA}
\end{gather*}
Тогда энергия основного состояния:
\begin{gather*}
     E = \dfrac{\hbar^2}{2mr^2} - \dfrac{e^2}{r} = -\dfrac{me^4}{2\hbar^2} = -13.6\text{ эВ}
\end{gather*}
Что интересно, эта оценка дает точно совпадающий с теорией Бора результат.\\
Часть 2. Квантовый гармонический осциллятор.\\
За этим страшным названием скрывается лишь обыкновенная двухатомная молекула. Для начала, почему и здесь тоже надо применять соотношение неопределенностей и почему эта штука не может обладать 0 энергией? Все просто -- если бы атомы в этой молекуле не колебались, это означало бы, что мы точно знаем и координату каждого из них и импульс (который неожиданно 0). Теперь собственно формулки. Полная энергия для такой молекулы, с учетом соотношения в форме Вейля:
\begin{gather*}
     E = \dfrac{p^2}{2\mu} + \dfrac{kx^2}{2} = \dfrac{\hbar^2}{8\mu x^2} + \dfrac{\mu \omega_0^2 x^2}{2}
\end{gather*}
Аналогично первой части найдем минимум энергии, но для простоты будем считать производную не просто по координате, а по квадрату координаты, так как минимумы будут совпадать:
\begin{gather*}
    \dfrac{dE}{dx^2} = -\dfrac{\hbar^4}{8\mu x^4} + \dfrac{\mu \omega_0^2}{2} = 0 \Rightarrow x^2 = \dfrac{\hbar}{2\mu\omega_0} \Rightarrow E = \dfrac{\hbar\omega_0}{2}
\end{gather*}
Этот результат оказывается также точно совпадает со строгим решением задачи о нулевом уровне энергии квантового гармонического осциллятора.
\subsection{Задача 2.31}
\label{task_2.31}
\paragraph{Условие}
Предполагая, что ядерные силы между нуклонами обусловлены обменом квантами ядерного поля -- виртуальными пионами, оценить радиус $\Delta r$ действия ядерных сил, если известно, что энергия покоя пионов $m_{\pi}c^2 \approx 140$ МэВ.
\paragraph{Решение}
Вот тут нам и пригодиться соотношение неопределенностей энергия-время. что это за пионы и о чем идет речь в задаче. Когда мы говорим о взаимодействии, мы рассматриваем этот процесс как обмен какими-то частицами. Как работает электромагнитное взаимодействие? Есть тело, которое заряжено. Оно испускает квант электромагнитного излучения, фотон, он летит к другому заряженному телу, которое его поглощает и тем самым узнает о первом теле. При этом второе тело также посылает фотон, которое поглощает первое тело. Это работает не только с электромагнитным, но и с другими взаимодействиями, в частности с сильным взаимодействием в атоме. А частицы такого взаимодействия называются пионы или $\pi$-мезоны.
Теперь, когда разобрались с тем, что происходит давайте запишем соотношение неопределенностей. Неопределенность по энергии это и есть масса нашего пиона, тогда неопределнность времени -- это время его жизни. А максимальная скорость с которой может частица перемещаться в нашим мире это скорость света. Отсюда и получим оценку на характерный радиус сильного взаимодействия:
\begin{gather*}
    R \le c\Delta t = c\dfrac{h}{\Delta E} = c\dfrac{h}{m_{\pi}c^2}\approx 2\cdot 10^{-13} \text{ см}
\end{gather*}


\subsection{Комментарии к задачам из задания}
\paragraph{Нулевки} В первой просто подставить в формулы, вторая по сути решена \ref{task_2.43}
\paragraph{Задача 2.10} Решена в задачнике
\paragraph{Задача 2.15} Решена, \ref{task_2.15}
\paragraph{Задача 2.26} Задачка ультра релятивистская, так что надо использовать формулу для полной энергии именно как корень, а дальше все получится
\paragraph{Задача 2.30} Часть решения я демонстрировал в теоретической части, а часть это просто оптика
\paragraph{Задача 2.38} Задача подозрительно напоминает часть 1 из \ref{task_2.43}
\paragraph{Задача 2.44} Решена в задачнике

\end{document}
