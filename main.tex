\documentclass[12pt]{article}
\usepackage[utf8]{inputenc}
\usepackage{amsmath,amssymb}
\usepackage{unicode-math}
\usepackage[T2A]{fontenc}
\usepackage[russian]{babel}
\usepackage{graphicx}
\usepackage{subfigure}
\usepackage{subcaption}
\usepackage{url}
\usepackage{float}


\DeclareGraphicsExtensions{.pdf,.png,.jpg}
\usepackage{hyperref}
\usepackage{wrapfig}
\usepackage[left=20mm, top=20mm, right=10mm, bottom=20mm]{geometry}

\usepackage{amsmath} 
\usepackage{amsfonts} 
\usepackage{amssymb} 
\usepackage{wasysym} 
\usepackage{fancyhdr}

\pagestyle{fancy}
\fancyhf{}
\lhead{Семинар }
\rhead{\textit{Клименок К.Л., МФТИ 2020}}
\rfoot{\thepage}



\begin{document} 
\title{\textbf{Семинар }}
\author{\textbf{Клименок Кирилл Леонидович}}
\date{18.09.2020}
\maketitle
\section{Теоретическая часть}
\subsection{}


\begin{figure}[h]
    \centering
    \includegraphics[width=\textwidth,height=\textheight,keepaspectratio]{}
    \caption{}
    \label{fig:sem_04}
\end{figure}


\section{Практическая часть}
\subsection{Задача}
\label{task_}
\paragraph{Условие}
\paragraph{Решение}

\subsection{Задача}
\label{task_}
\paragraph{Условие}
\paragraph{Решение}

\subsection{Задача}
\label{task_}
\paragraph{Условие}
\paragraph{Решение}

\subsection{Задача}
\label{task_}
\paragraph{Условие}
\paragraph{Решение}

\subsection{Задача}
\label{task_}
\paragraph{Условие}
\paragraph{Решение}

\subsection{Комментарии к задачам из задания}
\paragraph{Нулевки} 
\paragraph{Задача } 
\paragraph{Задача } 
\paragraph{Задача }
\paragraph{Задача }
\paragraph{Задача }
\paragraph{Задача }

\end{document}
